% Mallen hämtad från http://www.nada.kth.se/~serafim/xjobb/
% Finns även för svensk rapport

% useful stuff om Ex-jobb på KTH: https://www.kth.se/social/group/examensarbete-vid-cs/page/rapporten-5/

% Exempelrapporter, Kex: https://www.kth.se/social/course/DD143X/page/exempel-pa-rapport/

% HOW_TO_LATEX @michel:
%-------------------------------------------
% Nytt stycke: 2ggr Enter
% Ny rad: \\
% Kursiv text: \textit{*texten*}
% Fet text: \textbf{*texten*}
% Citationstecken görs med 2 st enkla, like so: ''text''
% Lägg in citering: \cite{*namn på referensen*}. Referensen måste läggas in i dokumentet example.bib under Meny->Project
% Punktlista: \begin{itemize} + "\item" för varje punkt + \end{itemize}
% Infoga figurer:
%\begin{figure}[H]
%\centering
%\includegraphics[width=14cm]{Overview.jpg}
%\caption{Overview of system interface\label{fig:overview}}
%\end{figure}
% referera till figur: \ref{*label*}
% För att kommentera ut block av text utan att radera använd "\begin{comment} + Texten + \end{comment}. Små kommentarer kan du använda "%", som du ser.
% För att lägga in synliga todo-kommentarer och flaggor använd "\todo[inline]{@Nick, vad ska det här betyda?}", eller vad du nu vill säga. [inline] är valfritt, utan det så hamnar det i marginalen i stället.
%

\documentclass[a4paper,11pt]{kth-mag}
\usepackage[T1]{fontenc}
\usepackage{cite}
\usepackage{pgfgantt}
\usepackage{textcomp}
\usepackage{url}
\usepackage{lmodern}
\usepackage{titlesec}
\usepackage{enumitem}
\setlist{nosep}
\usepackage{float}
\usepackage{graphicx}
\usepackage[linecolor=blue,backgroundcolor=blue!25]{todonotes}
\usepackage[utf8]{inputenc}
\usepackage[swedish,english]{babel}
\usepackage{modifications}
\let\cleartorecto\newpage
\let\cleardoublepage\clearpage
\renewcommand{\chapnamefont}    {\usefont{T1}{lmss}{sbc}{n}\boldmath\huge}

\title{ThesisTITLE}
%\subtitle{-  -}
\foreigntitle{TITEL på Svenska}
\author{Nick Nyman}
\date{May, 2020}
\blurb{Master's thesis in Industrial Management\\ Engineering degree project in Computer Science, ME200X\\KTH, Royal Institute of Technology\\Supervisors: Tomas Sörensson, KTH, Hannes Sahlin, Nordea\\Examiner: ??????}
\trita{TRITA xxx yyyy-nn}


\begin{document}
\addto\captionsenglish{\renewcommand{\chaptername}{}}
\renewcommand*{\chapternamenum}{}
\renewcommand*{\afterchapternum}{ }

\frontmatter
\pagestyle{empty}
\removepagenumbers
\maketitle
\todo[inline]{Kolla upp formatet på förstasidan samt bilden. Anpassa till indek ist för data. Infoga obligatoriskt försättsblad i pdf?}
\selectlanguage{english}
\begin{vplace}[0.5]
\begin{abstract}

\end{abstract}
\end{vplace}
\clearpage
\begin{vplace}[0.5]
\begin{foreignabstract}{swedish}

\end{foreignabstract}
\end{vplace}
\clearpage
\tableofcontents*
\mainmatter
\pagestyle{newchap}

\chapter{Introduction}



\section{Problem statement}



\section{Scope}

\section{Purpose}


\chapter{Background}



\chapter{Method}


\chapter{Results}


\chapter{Discussion}
This chapter contains the authors' discussion of both the results of the experiments and the methods used by which they were carried out.
\section{Discussion of results}


\section{Discussion of methods}


\chapter{Conclusion}


\bibliographystyle{unsrt}
\bibliography{resources.bib}
\newpage

\appendix
\addappheadtotoc

\end{document}
